\pagestyle{empty}

\begin{agradecimentos}

%ordem recomendada prelo Professor Valdex:

%1) Quem ofereceu a oportunidade de trabalho
Agradeço ao meu orientador Mirko Salomón Alva Sánchez e minha co-orientadora Viviane Rodrigues Botelho, pela oportunidade, pelos ensinamentos, pela paciência e pelas orientações durante minha jornada científica.

%2) Quem financiou o seu trabalho
Também gostaria de agradecer a CAPES, que realizou o financiamento da pesquisa e por também tornar possível todas as outras pesquisas no Brasil.

%3) Quem efetivamente contribuiu cientificamente durante a discussão do seu trabalho, professores do PPG
Agradeço aos Professores do PPGTIGSaúde que promoveram e promovem o programa em um ambiente tão acolhedor e tão humano, sempre com vigor, amor e paixão por ensinar.

%4) Quem efetivamente resolveu problemas da parte experimental do seu trabalho

%5) Os seus colegas de trabalho que contribuíram para a boa convivência no seu local de trabalho
Aos meus colegas de trabalho em inteligência artificial, o Felipe Ferreira de Freitas que me ensinou a resolver os problemas com o código, meu orientador Mirko Salomón Alva Sánchez por proporcionar um espaço de discussão em inteligência artificial, aos estudantes Fábio e Brenda pelas discussões sobre os temas que tanto possuem para agregar.

%6) Técnicos, secretários, e afins que contribuíram para a realização do seu trabalho
Também agradeço à Secretaria do programa, por tirar todas as dúvidas e por todos os problemas resolvidos, aos meus colegas de turma que se mostraram unidos nessa empreitada e à comissão coordenadora do programa, que sempre se mostrou extremamente eficiente.

%7) A sua família e membros presentes ou ausentes que merecem ser lembrados mas que não contribuíram fisicamente para a realização do seu trabalho no local de trabalho.
Por fim, porém não menos importante, agradeço à minha família, minha mãe Eni, meu pai Claiton e minha irmã Juliana, por sempre apoiarem minhas decisões e por estarem sempre presentes em minha vida, sem vocês nada disso teria sido possível.

\end{agradecimentos}
