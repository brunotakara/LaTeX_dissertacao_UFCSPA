%
% Documento: Introdução - colocar a label para conseguir chamar ela depois
%

\chapter{Introdução}\label{chap:introducao}

As metástases cerebrais múltiplas são os tumores intracranianos malignos mais comuns acometidos em adultos, com uma incidência de 20\% entre todos os pacientes com câncer, de acordo com \cite{Cho:2020}, \cite{Lin:2015} e \cite{Achrol:2019}. Esse tipo de neoplasia pode ser decorrente de outros tumores em outras localidades do corpo, uma vez que as células tumorais podem ser conduzidas via corrente sanguínea e sistema linfático para a o sistema nervoso central permite que tais células adentrem a barreira hemato encefálica, podendo assim dar início à formação da metástases \cite{Achrol:2019,Amsbaugh:2023}. Um estudo realizado em 2005, \cite{Shaw:2005} informa que o tempo de sobrevivência de pacientes diagnosticados com esse tipo de câncer era, em média, de 4 a 6 meses após tratamento radioterápico. Em menos de duas décadas depois, os avanços nas formas de diagnóstico e tratamento elevaram esse tempo para até 2 anos, dependendo do tipo de tumor, como mostra o estudo feito por \cite{Achrol:2019} que também indica que fatores prognósticos como sexo, idade e histórico patológico influenciam diretamente no tempo de sobrevida.

O tratamento de metástases cerebrais pode ser realizado através da ressecção cirúrgica, quando o número de metástases for pequeno e localizadas em regiões de fácil acesso, a radioterapia de cérebro inteiro (WBRT, do inglês \textit{Whole Brain Radiation Therapy}), quando o status de performance de Karnofsky, que mede o bem-estar do paciente pelo médico responsável, for baixo ou se a quantidade de metástases for muito alta. Já a radiocirurgia estereotáxica (SRS, do inglês \textit{Stereotactic Radiosurgery}) é recomendada quando o número de metástases estiver entre 1 e 4, podendo chegar a 10 em alguns casos, e o status de performance do paciente for normal \cite{Amsbaugh:2023}.

A recorrência metastática após a ressecção cirúrgica é relativamente alta, podendo chegar a 43\% em um período de 12 meses, o que pode ser melhor controlado aplicando-se as técnicas de radioterapia (RT) após a ressecção. No entanto, a aplicação dessas técnicas podem afetar negativamente algumas funções neurocognitivas \cite{Amsbaugh:2023}. 
 
A SRS é uma técnica de radioterapia que se provou com menor potencial de danos após o tratamento \cite{Tsao:2012}, podendo ser benéfica para o controle tumoral e para manter a função cognitiva dos neurônios, além de ser uma das técnicas mais utilizadas para tratamentos de neoplasias cranianas múltiplas, juntamente com craniotomia, quimioterapia ou WBRT.  O processo da SRS consiste na análise de imagens médicas pelo sistema de planejamento (TPS, do inglês \textit{Treatment Planning System}), onde serão elaborados as melhores condições da entrega  de uma alta dose de radiação aos volumes alvo, geralmente realizada em uma única entrega, e tendo em consideração as estruturas circunvizinhas às regiões irradiadas \cite{Bibault:2021}.

De acordo com a \textit{Internacional Comission on Radiation Units and Measurements} (ICRU) \cite{icru50:2015}, em um planejamento radioterápico, busca-se realizar o delineamento de uma região tumoral nas imagens em Ressonância Magnética (RM) ou Tomografia Computadorizada (TC). Em algumas situações é interessante utilizar os dois tipos de modalidades de imagens que são complementares uma à outra, pois a RM possui uma capacidade maior de diferenciar tecido nervoso de estrutura cerebral, porém está sujeita a um maior número de artefatos e distorções espaciais, enquanto que a TC possui uma excelente resolução espacial e mede diretamente as densidades eletrônicas para os cálculos de dose do planejamento radioterápico, porém não é capaz de contrastar tecido mole \cite{Hu:2019}.

Dentro do planejamento radioterápico, a região tumoral a ser delineada para tratamento é composta por quatro volumes de características distintas: o \textit{Gross Tumour Volume} (GTV), o \textit{Clinical Target Volume} (CTV), o \textit{Internal Target Volume} (ITV) e o \textit{Planning Target Volume} (PTV), sendo essa ordem indo do menor volume para o maior volume, com cada volume maior englobando o volume menor. O GTV é o volume do tumor a ser tratado, o CTV é o volume que engloba o tumor e considera patologias subclínicas em torno do GTV, o ITV leva em consideração a variação da posição espacial do CTV pela movimentação do paciente, por exemplo pela respiração ou diferença no posicionamento, adicionando uma margem interna (IM, do inglês \textit{Internal Margin}) ao CTV, e o PTV é o volume do planejamento considerando a margem de \textit{Setup} (SM, do inglês \textit{Setup Margin}) adicionada ao ITV, e que correspondente às variações dadas pelo equipamento ou geometrias utilizadas \cite{Mackie:2023}.

As definições para as margens IM e SM são padronizadas dependendo do tipo de órgão, do tipo de modalidade de imagem e do tipo da neoplasia a ser tratada, essas margens ser definidas com valores entre 1mm e 1,5 cm, sendo a SM levemente maior do que a IM \cite{Lee:2013}. Além desses volumes de tratamentos, outras regiões de interesse para o tratamento (ROIs, do inglês \textit{Regions of Interest}) são propostas, como os órgãos de risco (OARs, do inglês \textit{Organs at Risk}), que são delineados de modo a quantificar a absorção de dose de radiação por OARs saudáveis e responsáveis por funções vitais, tendo em vista que suas atividades fisiológicas podem sofrer alteração após o tratamento.

O \textit{Deep Learning} (DL) ou Aprendizado Profundo, é uma subárea do \textit{Machine Learning} (ML) ou Aprendizado de Máquina, que utiliza camadas sucessivas de operações matemáticas cujos valores são atualizados conforme é realizado o treinamento do modelo, as chamadas Redes Neurais Convolucionais (CNNs, do inglês \textit{Convolutional Neural Networks)} são um dos diversos tipos de redes neurais artificiais bastante utilizadas em Visão Computacional, pois utilizam operações de convoluções em sub-regiões de uma imagem, adquirindo características locais em imagens que podem ser 2D ou 3D \cite{Chollet:2017}.

A motivação do estudo se dá pela necessidade de uma forma automatizada de realizar o delineamento de metástases cerebrais e OARs, visto que esse é um trabalho extensivo e sujeito a variações entre radiologistas, a Inteligência Artificial (IA) pode auxiliar no delineamento como uma ferramenta de apoio à tomada de decisão do especialista. 

Para realizar a análise do delineamento das metástases cerebrais, este projeto de pesquisa propõe o emprego de uma CNN. Algumas pesquisas envolvendo a aplicação de \textit{Deep Learning} na radioterapia possuem bastante potencial e suas aplicações podem ser feitas em qualquer parte do planejamento radioterápico \cite{Siddique:2020}. Para isso, as imagens obtidas no delineamento do TV e dos OARs serão utilizadas como entrada para o treinamento de diferentes CNNs de modo a avaliarmos qual a melhor arquitetura para ser utilizada, alguns exemplos são a \textit{Visual Geometry Group Network} (VGGNet), a \textit{GoogLeNet} e as \textit{Residual Neural Networks} (ResNets), como a ResNet-101 que tem tido resultados promissores para a predição de histogramas dose-volume (DVHs, do inglês \textit{Dose Volume Histograms}), conforme aponta \cite{Chen:2021} que obtiveram valores para a distribuição de dose similares aos valores obtidos pelo \textit{software} do planejamento radioterápico RapidPlan$^{\text{TM}}$, e também em \cite{Chen:2018} onde usam a arquitetura ResNet-101 para gerar DVHs em casos de câncer de nasofaringe de maneira automatizada e personalizada para cada paciente. Já \cite{Fan:2018} aponta para a utilização da ResNet-50 na automatização do planejamento radioterápico com base na distribuição volumétrica de dose realizada por técnicas de DL. Nesses estudos, os dados obtidos foram analisados através da distribuição de dose nos volumes de acordo com as restrições de dose ideais, e a métrica mais recorrente utilizada para essa avaliação foi a de Erro Absoluto Médio (MAE, do inglês \textit{Mean Absolute Error}), utilizando o valor-p para interpretação em cada ROI estudada.

Em relação à função de segmentação realizada por uma rede neural, alguns estudos apontam para a dificuldade em se detectar metástases pequenas \cite{Grovik:2020,Bibault:2021,Charron:2018} quando o modelo também é treinado para metástases maiores, o que pode dificultar a definição de um modelo mais geral. Alguns treinos realizados apenas com metástases menores melhoram a eficiência de detecção, mas precisam passar por refinamentos e seus resultados não são satisfatórios quando comparados aos resultados obtidos para metástases maiores \cite{Bousabarah:2020,Zhou:2020}. Com isso em vista, a segmentação de metástases pequenas é um ramo com potencial para exploração.

Espera-se que com os resultados obtidos neste estudo os tratamentos utilizando radioterapia tenham uma ferramenta embasada em \textit{Deep Learning} para auxiliar as equipes médicas durante o planejamento radioterápico, com ênfase na etapa de identificação e segmentação das regiões metastáticas, principalmente aquelas menores e de difícil identificação devido ao tamanho. Em muitos casos, o processo de delineamento é demorado, podendo ser acelerado com a utilização de \textit{Machine Learning}.

\section{Questão da Pesquisa}

Como a utilização de \textit{Deep Learning} pode auxiliar no delineamento automático de metástases cerebrais?

\section{Objetivos Geral e Específicos}

\subsection{Objetivo Geral}

Utilizar modelos de aprendizado profundo como uma ferramenta para automatizar o delineamento de metástases cerebrais múltiplas através do uso de imagens de ressonância magnética e tomografia computadorizada.

\subsection{Objetivos Específicos}

\begin{enumerate}[I]
    \item Buscar \textit{datasets} públicos em RM e TC 
    \item Avaliar arquiteturas CNN para escolha daquela com melhores resultados na identificação e segmentação de metástases cerebrais.
    \item Verificar a concordância da segmentação das regiões de interesse (ROIs) realizada pelas redes neurais em relação aos dados anotados por profissionais.
\end{enumerate}
