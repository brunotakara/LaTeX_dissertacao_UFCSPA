%
\chapter{Trabalhos Relacionados}\label{relacionados}

A revisão sistemática com meta-análise de \cite{Cho:2020} aponta para a escolha de técnicas em \textit{Deep Learning} para a detecção de metástases cerebrais, que se mostrou superior às técnicas clássicas de \textit{Machine Learning}. Essas técnicas clássicas têm sido utilizadas há mais de 20 anos e ainda aparecem em literatura sob a nomenclatura \textit{Computer-aided Detection} (CAD) ou \textit{Computer-aided Diagnosis} (CADx), porém estão caindo em desuso devido à alta taxa falsos positivos. A utilidade do DL aplicado na RT é diversa, tendo funções como o aprimoramento da reconstrução da imagem, remoção de ruídos, segmentação automática de estruturas, adequação às recomendações de delineamento, dosimetria, previsão de resposta ao tratamento, previsão de toxicidades, entre outros \cite{Bibault:2021}. 

Sendo assim, foi realizada uma busca nas bases de dados PubMed, IEEE Xplore e Epistemonikos para a busca de publicações relacionadas com a pesquisa de \textit{Deep Learning} na tarefa de segmentação de metástases cerebrais. A busca utilizando os descritores (brain AND "deep learning" AND segmentation AND metas*) retornou um total de 72 artigos nessas três bases, dos quais foram excluídos 5 duplicatas, 6 artigos de revisão e 19 artigos que não contemplavam o tema. Além disso, foram adicionados dois artigos \cite{Hu:2019} e \cite{Rudie:2021} de outras fontes. Foram selecionados 31 dos 44 artigos restantes para a leitura do texto completo, e a distribuição dessas publicações ao longo dos anos pode ser vista na Figura \ref{fig:anos}. Os motivos da exclusão dos 13 artigos foram porque não trataram da segmentação de metástases ou realizaram um treino para segmentação de tumores de modo geral, sem especificar o treino para o caso de metástases. Na Figura \ref{fig:locais}, estão destacados os países envolvidos nas pesquisas selecionadas, e uma comparação entre os resultados destes estudos pode ser visto na Tabela \ref{tab:literatura}. A seguir, encontra-se uma descrição com a síntese dos métodos e resultados dos trabalhos incluídos na revisão.

\begin{figure}[!htb]
\centering
    \includegraphics[width=0.9\textwidth]{./04-figuras/anos2.png}
	\caption{Publicações ao longo dos anos dos artigos encontrados nas bases PubMed, IEEE Xplore e Epistemonikos que foram selecionados para a leitura completa}\vspace{-0.2cm}
    \fonte{Autoria própria.}
    \label{fig:anos}
\end{figure}

\begin{figure}[!htb]
\centering
    \includegraphics[width=0.9\textwidth]{./04-figuras/locais3.png}
	\caption{Quantidade das publicações incluídas na revisão por país}\vspace{-0.2cm}
    \fonte{Autoria própria.}
    \label{fig:locais}
\end{figure}

\cite{Grovik:2020} utilizaram o mesmo dataset deste estudo, o \textit{BrainMetShare}, para realizar a identificação das metástases cerebrais com o uso de redes neurais com arquitetura de rede 2.5D baseada na GoogLeNet, e avaliaram as métricas de precisão, sensibilidade, F1 Score, coeficiente Dice e AuROCC. Entre os resultados obtidos, obtiveram valores para a AuROCC de $0,98 \pm 0,04$ e, para precisão, sensibilidade e coeficiente Dice, encontraram os valores de, respectivamente, $0,79 \pm 0,20$, $0,52 \pm 0,22$ e $0,79 \pm 0,12$.

Um outro estudo realizado por \cite{Hu:2019} aplica uma função perda com pesos diferentes de acordo com o volume da metástase cerebral, discorrendo sobre o uso da função perda Dice ser ideal para conjuntos de dados com metástases de tamanhos semelhantes, o que pode trazer resultados superestimados quando o modelo é aplicado em casos com metástases de tamanhos variados. Isso pode ser um problema, já que é comum casos em que coexistem metástases de tamanhos diversos. Esse risco se mostra ainda mais relevante ao considerar-se que a probabilidade de um radiologista encontrar uma metástase é menor conforme menor for o seu tamanho. Nesse estudo, \cite{Hu:2019} utilizaram um conjunto de dados formado por imagens de RM e imagens de TC, visto que cada técnica possui suas vantagens e desvantagens, e a combinação de ambas possibilita uma maior confiabilidade diagnóstica. Para o treino, foi utilizada um \textit{ensemble} de duas arquiteturas, a 3D U-Net e a DeepMedic. Para o treino da rede 3D U-Net, foi utilizado o otimizador {\ttfamily RMSprop}, com LR de $10^{-3}$, BS de 1 e durante 300 épocas. Já para a rede DeepMedic, foram utilizados três caminhos paralelos, um que treina a rede com as imagens em sua máxima resolução e outro que utiliza \textit{patches}, ou pedaços, que são subamostras dessas imagens para o treino. Os resultados obtidos se mostraram promissores quando aplicado o \textit{ensemble} dessas duas redes, alcançando um DSC de $0.740 \pm 0,022$, uma precisão de $0,779 \pm 0,010$ e uma sensibilidade de $0,803 \pm 0,001$. Assim, os autores encontraram uma maior eficiência da rede em detectar metástases pequenas ao utilizar a função perda dependente do volume da metástase, que chamaram de \textit{volume-aware Dice loss}.


\cite{Rudie:2021} realizaram um estudo retrospectivo com 413 pacientes que foram submetidos à radiocirurgia. Foram utilizadas 563 imagens de ressonância magnética ponderadas em T1 pós contraste, separadas em 413 para treino, 50 para validação e 100 para teste. Em uma etapa do pré-processamento, foi realizada a subtração da imagem ponderada em T1 pré-contraste da imagem ponderada em T1 pós-contraste, de modo a ter um maior enfoque nas regiões de acúmulo de contraste. O treino foi realizado utilizando uma rede 3D U-net, com regularizador de normalização por \textit{batch} (BN, do inglês \textit{Batch Normalization}), otimizador Adam e com \textit{learning rate} de $10^{-4}$. A avaliação foi feita pelo coeficiente Dice, precisão e sensibilidade, que resultaram em valores de, respectivamente, 0,75, 91,5\% e 70,0\%. Para metástases com tamanhos maiores do que 6 milímetros, a sensibilidade da rede foi de 96,4\%, indicando uma correlação entre o tamanho da metástase e a capacidade da rede em identificá-la, assim como aponta \cite{Hu:2019}. Além disso, os autores, avaliaram a confiabilidade do formato da metástase encontrada pela rede através do cálculo da distância de Hausdorff. Por fim, as mesmas métricas foram avaliadas entre dois diferentes radiologistas com 4 e 5 anos de experiência, que levaram cerca de 15 a 20 minutos para realizar a segmentação e obtiveram DCS de 0,85, sensibilidade de 87,9\% no geral e 98,4\% para metástases maiores do que 6 mm, superando os resultados encontrados pela rede neural, que levou cerca de 45 segundos para realizar a mesma tarefa.

Algumas considerações feitas pelo estudo de \cite{Liu:2017} apontam para a dificuldade de realizar a segmentação de metástases pequenas, cujo tratamento é realizado por meio de SRS, bem como a formação de agrupamentos de metástases podem dificultar o delineamento automático, além de relatar que é comum uma única aquisição de imagens em RM ponderadas em T1 com contraste, dificultando a aplicação de redes neurais treinadas com conjuntos de dados em outras ponderações ou sequências de pulso. Nesse estudo, os autores utilizaram a arquitetura de rede DeepMedic com o acréscimo de um caminho paralelo adicional, que realizou uma subamostragem de menor resolução para encontrar características globais, com um \textit{kernel} de convolução maior do que o presente nos outros caminhos paralelos. Em relação ao conjunto de dados, foram utilizados dois conjuntos, o primeiro foi o \textit{dataset} Brain Tumor Segmentation (BraTS) \cite{BRATS:2015}, com imagens em RM com contraste ponderadas em T1 e em FLAIR, para um total de 240 pacientes, e o segundo foi um conjunto de imagens de ressonância em T1 com contraste, provenientes do \textit{University of Texas Southwestern Medical Center}. Para o treino, foi utilizado um \textit{batch size} de 10 imagens durante 35 épocas para a validação com o BRATS e durante 15 épocas para as imagens obtidas do centro médico. Nota-se que para a validação, o \textit{batch size} foi alterado para 48. Os autores encontraram um DCS de $0.67 \pm 0,03$ e uma AuRROC de $0,98 \pm 0,01$, apontando para uma segmentação aceitável de metástases maiores do que 1,5 cm$^3$. Assim como \cite{Hu:2019,Rudie:2021} a eficiência da segmentação decai consideravelmente conforme menores os tamanhos das metástases.

A utilização de imagens em RM multimodal foi estudada por \cite{Charron:2018} na tarefa de segmentação de metástases cerebrais, em que foi utilizado um \textit{dataset} de imagens ponderadas em 3D T1 CC, FLAIR, e 2D T1, para um número de 182 pacientes, em um treino dividido em 80\% para treino, 10\% para validação e 10\% para teste. Um mapa da distribuição espacial da ocorrência de metástases indicou maior ocorrência nas regiões do cerebelo e lobo frontal. Além disso, foram criados novos casos de 1200 pacientes virtuais para o teste. Os melhores hiperparâmetros para a realização do treino foram investigados, resultando em um BS de 10, com amostras de tamanho de $24\times24\times24$ mm$^2$, durante 30 épocas, com 20 sub-épocas, para 1000 segmentos e realizado para 50 pacientes por época. As métricas de avaliação utilizadas foram a sensibilidade e o número de falsos positivos para a tarefa de detecção. Para a tarefa de segmentação a métrica utilizada foi o DSC, a sensibilidade e a precisão. Os autores apontam para o cuidado em se realizar o \textit{downsampling} das imagens, uma vez que a compressão dos voxels podem truncar algumas informações radiológicas, como os vasos sanguíneos que aparecem hiperintensos na imagem, de modo que a rede neural os confunda com regiões metastáticas, resultando em um maior número de falsos positivos. Um outro achado do estudo foi a utilização de \textit{batch sizes} pequenos para encontrar melhores valores para DSC, cuja razão é devida à inomogeneidade do \textit{dataset} e à menor probabilidade de se ter uma alta variedade de tamanhos de tumores em um menor conjunto de dados, o que permite o treino de um caminho neural adaptado para um tipo de metástase em vez de buscar encontrar um caminho global que generalize o resultado para qualquer tamanho de metástase, que é o que aconteceria em um \textit{batch size} maior, e que geralmente é utilizado em dados mais homogêneos. Entre as descobertas do estudo, destaca-se a utilização multimodal de imagens para o treino da rede, que resultou em uma sensibilidade de 98\% para a identificação de metástases, enquanto que o acréscimo de pacientes virtuais não ocasionou em melhorias significativas.

\cite{Dikici:2020} utilizaram uma abordagem de seleção através do Laplaciano da Gaussiana, utilizando essa aproximação ao determinar possíveis candidatos para serem inseridos no modelo de segmentação e, após essa seleção, foram aplicadas deformações elásticas aleatórias e correções gama aleatórias nas imagens para \textit{data augmentation}. A rede utilizada foi a CropNet, que tem como \textit{input} regiões volumétricas isotrópicas de tamanho de 1 mm$^3$ para cada \textit{voxel}, e o resultado da rede para a sensibilidade foi de 0,90, com uma média de falsos positivos de 9,12 por paciente.

Com o uso de uma rede com base em uma 3D \textit{Fully Connected Netword}, \cite{Xue:2020} propuseram a rede BMDS Net para a segmentação de metástases cerebrais, que possui dois estágios, o primeiro para identificação e o segundo para segmentação. Para a identificação, a imagem original de tamanho $256\times256\times120$ \textit{pixels} foi diminuída em quatro vezes o tamanho original e, após a detecção, a imagem era redimensionada para seu tamanho original para realizar a segmentação, que consistia no isolamento da região detectada do resto da imagem, através do uso de \textit{bounding-boxes} que seriam os \textit{inputs} da rede de segmentação. Os hiperparâmetros utilizados no treino incluíram uma LR inicial de 0,002 com decaimento de 0,1 por 500 épocas, o valor de \textit{momentum} igual a 0,9 e foi utilizado o inicializador Xavier. O resultado obtido para a acurácia de detecção foi de 100\%, e para a segmentação os valores de DSC, sensibilidade e especificidade foram de, respectivamente, $0,85 \pm 0,08$, $0,96 \pm 0,03$ e $0,99 \pm 0,0002$.

\cite{Bousabarah:2020} realizaram modificações em uma U-Net para o treino de metástases separadas por tamanho, e compararam os resultados com uma U-Net convencional. Após os treinos, foi comparado os métodos de \textit{ensemble} por soma e por voto de maioria entre as três redes, que foram treinadas por 450 épocas, com LR inicial de 0,0001 e taxa de decaimento de 1\% por época, otimizador Adam com regularização L2, BS de 2 e validação cruzada em 4 \textit{folds}. Os autores concluíram que o uso de \textit{ensemble} por soma é melhor para as métricas de sensibilidade (0,82) e DSC (0,74), enquanto que o \textit{ensemble} por voto de maioria apresenta melhores precisão (0,96), taxa de falsos positivos (0,08), \textit{F1-Score} (0,85), e DSC médio (0,71). Além disso, a rede treinada apenas com metástases cerebrais pequenas foi até 50\% melhor na segmentação de metástases pequenas quando comparada às redes treinadas com todos os tamanhos de metástases.

\cite{Zhou:2020} propuseram uma rede chamada MetNet em dois estágios, sendo o primeiro uma rede \textit{single-shot detector} (SSD) e o segundo uma rede FCN. A rede SSD foi responsável pela identificação da metástase enquanto a rede FCN foi utilizada para a segmentação 2D das metástases em uma aplicação parecida com a realizada por \cite{Xue:2020}, que realizou a mesma estratégia em três dimensões. No estudo, também foram investigadas três variações da função perda de Tversky e uma Dice e os \textit{inputs} da rede de segmentação possuíam tamanho de $64\times64$ \textit{pixels}, outros hiperparâmetros do treino não foram informados pelos autores. Os resultados gerais para a precisão, sensibilidade, FPP e DSC foram, respectivamente, $0,58 \pm 0,25$, $0,88 \pm 0,19$, $3 \pm 3$ e $0,85 \pm 0,13$. Os autores também realizaram testes para metástases de tamanhos distintos, cujos resultados foram, em alguns casos, mais do que 2 vezes melhores para as metástases de tamanhos maiores do que 6 mm quando comparados com os resultados para metástases de tamanhos menores do que 3 mm.

O uso da rede DeepMedic 3D por \cite{Pennig:2021} com \textit{inputs} de tamanho $25\times25\times25$ \textit{pixels} formados por \textit{patches} constituintes por regiões cerebrais com e sem metástases balanceados em 50\% cada, trouxe resultados para DCS de 0,75, precisão de 0,74, sensibilidade de 0,88 e falsos positivos por paciente de 0,71. Os hiperparâmetros de treino incluíram BS de 10, durante 35 épocas com função perda do tipo \textit{Dice loss}, a LR teve decaimento de 50\% com paciência de 3 épocas e seu valor inicial não foi informado. A DeepMedic com \textit{patches} de mesmo tamanho e mesmos hiperparâmetros de rede também foi utilizada por \cite{Jünger:2021} para a segmentação de metástases, chegando em resultados de 0,72 para o DCS, 0,851 para a sensibilidade e 1,8 para a FPP, a única diferença em relação ao estudo de \cite{Pennig:2021} foi o número de pacientes, que antes era 69 e aumentou para 98.

\cite{Park:2021} treinaram uma rede 3D U-Net com \textit{inputs} de tamanho $184\times184\times152$ \textit{pixels} e combinaram dois modos de aquisição de imagem em RM, Gradiente Eco e Sangue Escuro, para comparar com os treinos realizados apenas com um dos modos. Foi realizada validação cruzada em 5 \textit{folds} com otimizador Adam, durante 500 épocas com LR de 0,0001. Com exceção da métrica de precisão para todas as metástases, o modelo utilizando a combinação dos dois modos de aquisição de imagem foi melhor, alcançando sensibilidade de 0,907, precisão de 0,931 e DSC de $0,822 \pm 0,152$.

\cite{Yi:2021} investigaram a performance de uma rede 2.5D DeepLabv3 ao remover uma das sequências de pulso das imagens de RM do \textit{input}, que constou com imagens 2d de tamanho de $256\times256$ \textit{pixels} para quatro tipos de sequências diferentes das quais foram incluídas 5 \textit{slices} de cada no canal do \textit{input}. Cada sequência de pulso foi removida aleatoriamente, e os valores dos 5 \textit{slices} foram definidos como 0 para o treino. A duração do treino foi de 10 épocas e outros hiperparâmetros não foram informados pelos autores. O resultado obtido para a DSC foi de 0,85, enquanto que para precisão e sensibilidade foram, respectivamente, 0,68 e 0,92.

O estudo realizado por \cite{Yoo:2021} utilizou redes U-Net 2.5D e 3D na tarefa de segmentar metástases cerebrais, alcançando um DCS, precisão e sensibilidade de, respectivamente, 0,62, 0,75 e 0,74. Os autores realizaram uma análise de acordo com o número de metástases e de acordo com o volume total de metástases por paciente, e concluíram que a rede 3D é melhor para a segmentação de metástases pequenas, enquanto que a rede 2.5D é superior para a segmentação de metástases maiores. Além disso, foi comparada a performance das redes U-Net com redes \textit{Fully Convolutional One-Stage Monocular} 3D, e ambas as redes U-Net foram superiores na tarefa de segmentação. O treino da rede U-Net 2.5D foi realizado com o otimizador Adam, com regularização por decaimento de pesos desacoplados, tamanho de \textit{mini-batch} igual a 100 e LR com valor de 0,001. Além disso foram utilizados \textit{Dropouts} nas camadas do \textit{encoder} e \textit{decoder} com probabilidade de 0,2 e a função perda utilizada nos \textit{outputs} do \textit{decoder} foi a \textit{cross-entropy}. Por outro lado, o treino da rede U-Net 3D contou com a subamostragem em \textit{patches} formados por \textit{voxels} de tamanho de $96\times96\times96$, sendo amostrados 8 voxels por volume tanto para regiões com lesão quanto para regiões sem lesão, totalizando 40\% dos voxels obtidos com lesão e 60\% dos voxels sem lesão. O \textit{dropout} utilizado também foi de 0,2 com a função perda de Dice binária, otimizador Adam, LR de 0,005 e tamanho de \textit{mini-batch} igual a 30.

\cite{Grøvik:2021} realizou um novo estudo utilizado uma rede baseada na DeepLabv3, nomeada \textit{Input-Level Dropout}, para a segmentação de metástases cerebrais com a remoção de uma das quatro sequências de pulso obtidas em RM, como uma maneira de lidar com dados faltantes no dataset. Foram utilizados cinco \textit{slices} de cada modalidade de imagem para o treino, que foram acumulados no canal de cor do \textit{input array} que continha dimensão de $256 \times 256 \times 20$. O chamado \textit{Input-Level Dropout} foi aplicado nos 5 \textit{slices} de uma determinada sequência de pulso, cujos valores do tensor foram definidos como 0's. Os resultados obtidos com essa rede para o coeficiente Dice ($0,795 \pm 0,105)$, a IoU ($0,561 \pm 0,225)$, a sensibilidade ($0,671 \pm 0,262$) e precisão ($0,790 \pm 0,158$) foram superiores quando comparado com a rede DeepLabv3 original, indicando a relevância do método aplicado. 

No estudo feito por \cite{Hsu:2021} em imagens tanto de RM quanto de TC para a identificação e segmentação de metástases cerebrais, foi utilizada uma rede 3D V-Net, arquitetura parecida com a U-Net, com convoluções de \textit{kernel} $3\times3\times3$ tanto na parte de \textit{downsawmpling} quanto na parte de \textit{upsampling}, com \textit{stride} de 2 e função de ativação \textit{parametric rectified linear unit} (PReLU), com \textit{batch normalisation} após cada camada de convolução e pesos dos \textit{kernels} regularizados com o iniciador Xavier. Para o treino em questão, o volume de \textit{input} foi reduzido para 48 mm$^3$ e o número de \textit{feature channels} foi aumentado para 24, bem como o número dos \textit{strides} que passaram a ser 3, 3, 3, 3 e 4 conforme aumentava-se o \textit{downsampling}, otimizador Adam com LR de 0,001 com \textit{Early stopping}. Além disso, foi utilizado \textit{data augmentation} e transformações nas imagens envolvendo a equalização do histograma com base em outro paciente aleatório, deformação \textit{B-Spline} aleatória e rotação 3D aleatória de até 20º. Os autores apontaram para o uso da função perda \textit{soft} Dice, uma vez que os dados estavam muito desbalanceados com menos do que 0,1\% dos voxels contendo metástases, e indicaram que nesse cenário a função perda \textit{cross-entropy} não deveria ser utilizada. 

A utilização de uma mistura de redes U-Net 2D e 3D também foi utilizada por \cite{Cho:2021} para a detecção e segmentação de metástases cerebrais. Nesse modelo, a rede 3D foi usada para a identificação e aumento da dimensionalidade das metástases identificadas para que a rede 2D fizesse a segmentação. Além disso, o treino foi dividido em três tipos de validação, a primeira com pacientes deslocados temporalmente entre os anos de 2015 e 2016, a segunda com pacientes de outra instituição e a terceira com pacientes dos anos entre 2017 e 2020, com deslocamento temporal maior do que o primeiro conjunto. Dessa forma, as métricas de avaliação foram melhores para os pacientes do terceiro conjunto, com valores de sensibilidade de 0,947, a quantidade de falsos positivos encontrada de 0,5 por paciente e o DCS de $0,82 \pm 0,20$. Vale notar que a análise foi realizada para metástases de tamanhos maiores do que 5 mm e os parâmetros utilizados para o treino da rede de identificação, a 3D U-Net, incluíram a função perda \textit{Dice Loss} para \textit{inputs} de tamanho $192\times192\times192$ com otimizador Adam, BS de 1 e LR com valor inicial de 0,001 durante 300 épocas. Nota-se que entre as épocas 100 e 250, a LR utilizada foi de 0,0001 e após a época 250 a LR foi de 0,00001. Já para o treino da rede de segmentação, a 2D U-Net, os parâmetros utilizados foram \textit{input} de $512\times512$ pixels com \textit{data augmentation} de 16 vezes, e os outros hiperparâmetros não foram mencionados na publicação.

\cite{Sun:2021} realizaram uma comparação entre os resultados obtidos entre as redes U-Net 2D e 3D para a segmentação de metástases cerebrais em um banco de dados próprio de 450 pacientes. As imagens utilizadas passaram por \textit{data augmentation} e regularização L2 para as duas arquiteturas. A métrica estudada foi a AuROCC que se mostrou maior para a rede 3D com regularização para todos os tamanhos de metástases (0,876), seguida da rede 2D com regularização (0,815). A rede sem regularização 3D foi melhor para a segmentação de metástases com volume acima de 1,37 cm$^3$ (AuROCC = 0,905) quando comparada com a rede sem regularização 2D U-Net (AuROCC = 0,88), essa que se mostrou superior para a segmentação de metástases menores (AuROCC = 0,759) contra a AuROCC de 0,714 da rede 3D. Dessa maneira, aponta-se para a importância da técnica de regularização L2 para uma melhor AuROCC na tarefa de segmentação de metástases cerebrais, bem como a relevância das características espaciais obtidas pela rede 3D para uma melhor segmentação.

Uma alteração na arquitetura DeepMedic proposta por \cite{Huang:2022} aumentou significativamente as métricas de sensibilidade, de 0,853 para 0,975 e de precisão, de 0,691 para 0,987. A alteração foi o uso de uma função perda específica para o caso de metástases cerebrais, chamada de \textit{Volume-level Sensitivity-Specificity} (VSS), e foi empregada no lugar da \textit{Binary Cross-Entropy} para a tarefa de identificaçao. Essa função perda foi criada para mitigar a variabilidade dos resultados obtidos para metástases de tamanhos diversos, uma vez que a função perda para metástases pequenas não possui grande impacto no conjunto todo. Assim, foi realizada o \textit{ensemble} das duas funções perdas para a tarefa de segmentação, que contou com um número de 60 épocas, com LR inicial de 0,001 e otimizador \textit{Root Mean Squared Propagation} (RMSProp) com momento de Nesterov ($m=0,6$, $\rho=0,9$ e $\epsilon = 0,0001$). Os dados de input possuíam tamanho de $37\times37\times37$ com 9 camadas de convolução de \textit{kernels} de tamanho $3\times3\times3$ sem \textit{padding}, de modo que o \textit{output} do modelo tinha o tamanho de $19\times19\times19$ \textit{pixels}, e os segmentos passaram por \textit{data augmentation} envolvendo mudança aleatória de intensidade, inversão e rotação.

Para a segmentação de metástases pequenas, com volumes menores do que 0,04 cm$^3$, uma rede foi proposta por \cite{Yoo:2022} com base nas redes U-Net 2.5D e 2D. As imagens de tamanho original de $1024 \times 1024$ \textit{pixels} foram recortadas em \textit{patches} de $128 \times 128$ \textit{pixels} com sobreposição de \textit{patches} e um espaçamento de 64 \textit{pixels} entre as sobreposições. Técnicas de \textit{data augmentation} incluíram inversão vertical e horizontal, bem como rotação e borramento aleatórios. A rede U-Net teve alteração na função de ativação da habitual ReLU para a \textit{Exponential Linear Unit} (ELU), para evitar o gradiente nulo, com \textit{Batch Normalisation} antes da função de ativação ELU. A LR inicial foi de 0,0001 com redução de 50\% para cada vez que a métrica acompanhada não mudasse em duas épocas consecutivas, em um total de 30 épocas, com BS de 256, otimizador Adam e regularização L2 ($\beta = 0,00001$). Os resultados obtidos para a segmentação teve uma sensibilidade de 0,966, com coeficiente Dice de 0,55 e um número médio de FP de 1,25 por paciente. Vale ressaltar que o conjunto de dados utilizados consistia majoritariamente de metástases de tamanhos menores do que 0,1 cm$^3$, e a discussão desse trabalho é relevante para o estudo de metástases de tamanhos pequenos, cuja identificação é um desafio para as pesquisas realizadas nessa área.

Para a realização independente de segmentação de quatro tipos de tumores cerebrais: glioblastomas, gliomas de baixo grau, meningiomas e metástases, \cite{Bouget:2022} utilizaram uma arquitetura de rede com base na AGU-Net e analisaram a correlação de 25 métricas distintas para avaliar a performance dos modelos, apontando para a relevância das métricas de diferença de volume e distâncias de Hausdorff, Mahalanobis e de superfície simétrica média do objeto. Em relação à rede, foi utilizada a função perda \textit{class-averaged Dice loss}, com otimizador Adam e BS de 32, paciência de 30 épocas para \textit{early stopping} e LR de 0,001, ou LR de 0,0001 quando aplicado \textit{transfer learning} para um \textit{fine-tuning} do melhor modelo encontrado para a segmentação de glioblastomas. Foi utilizado \textit{data augmentation} com probabilidade de 50\% inversão vertical e horizontal, rotação aleatória de 20º e translação de até 10\% nos eixos. A eficiência da rede se mostrou melhor para a segmentação de metástases e glioblastomas para todas as métricas analisadas. Para metástases, a rede alcançou um valor de $0,8773 \pm 0,1894$ para o DSC e $0,8156 \pm 0,2042$ para IoU, sendo ambas as melhores métricas encontradas entre os diferentes tipos de tumor.

\cite{Liang:2022} estudaram diversas variantes de 3D-Unets para a tarefa de segmentação de metástases cerebrais, e relatam que um número de \textit{kernels} de 128 com tamanho de \textit{input} de $64\times64\times64$ \textit{voxels} demonstrou os melhores resultados em uma validação cruzada de 5-\textit{folds}. No conjunto de teste que contava com 327 metástases, o modelo alcançou um DSC de 0,73, com sensibilidade de 0,91 e taxa de falsos positivos de 1,7, e indicam que o modelo possui melhor performance para a segmentação de metástases que possuem mais que 12 mm de diâmetro.

\cite{Lyu:2022} realizaram a segmentação de metástases com uma variante da U-Net, e compararam seus resultados com as redes U-Net, AttU-Net e U-Net++, e encontraram melhores resultados para o DSC ($0,878 \pm 0,297$), sensibilidade ($0,932 \pm 0,363$) e especificidade ($0,999 \pm 0,004$) com o modelo que propuseram. O modelo foi treinado com duas modalidades de imagens de ressonância magnética, e foi utilizada uma rede cycleGAN para gerar as imagens faltantes em caso de haver apenas aquisição em uma das modalidades. Foi utilizado o otimizador Adam, com LR de 0,0001, e BS de 12, sendo aplicada a função de ativação ReLU nas camadas de convolução. A \textit{callback} utilizada foi a de \textit{loss decay} com margem de variação de pelo menos 1\% após 20 épocas consecutivas, e a função perda analisada foi uma combinação da \textit{cross-entropy} com a Dice \textit{loss}.


Uma outra pesquisa envolvendo o \textit{BrainMetShare}, em conjunto com o BrATS e um \textit{dataset} adquirido localmente, foi realizada por \cite{Liew:2022} para a segmentação de metástases cerebrais. Eles utilizaram uma arquitetura de rede com base em uma N-Net, que é uma versão modificada da U-Net, e fizeram algumas alterações nas conexões entre as camadas, como a adição de blocos não-locais. O treino foi realizado com \textit{patches} de $96 \times 96 \times 96$ \textit{pixels} com janela de deslocamento de 32 \textit{pixels}, um BS de 4, uma LR adaptativa com valor inicial de 0,001 para as 75 primeiras épocas, aumentando para 0,1 nas 45 épocas seguintes, e 0,01 nas 30 épocas finais, totalizando 150 épocas. A função perda utilizada foi uma combinação da \textit{mean-squared error} e a \textit{binary cross-entropy}. Após realizarem o treino para os três \textit{datasets}, foi aplicado um \textit{framework} iterativo para o \textit{self-training} gradual do modelo, denominado CAVEAT, os resultados obtidos para a rede foram otimizados, e incluíram uma sensibilidade de 0,740 para o \textit{dataset} local, e 0,811 para o \textit{BrainMetShare}, sendo que a quantidade de falsos positivos encontrados por paciente foi de 3,131 e 2,952 para esses mesmos \textit{datasets}, respectivamente.

Ao utilizarem uma combinação de funções perda Dice, \textit{boundary} e \textit{Volume-Aware}, \cite{Chartrand:2022} investigaram as diferentes combinações nos resultados para a segmentação de metástases cerebrais utilizando uma rede de arquitetura 3D U-Net. Para o treino, foram utilizados \textit{patches} de tamanho de $128\times128\times128$ \textit{pixels}, com uso do otimizador Adam, com LR de 0,001 que era dividida pela metade a cada 50 épocas sem variações, durante um total de 400 épocas. Em relação às transformações nas imagens, foram aplicadas rotações aleatórias, ruído gaussiano e correção gama de contraste. Os valores encontrados para a sensibilidade foram de 0,909, com DSC de 0,73 e 0,66 falsos positivos por paciente. Os autores também apontaram para a piora dos resultados das métricas conforme o tamanho das metástases era menor e indicam a melhora da segmentação quando utilizada uma combinação de função perda \textit{Dice} com \textit{Volume-Aware sampling}, que realiza a amostragem com base em uma probabilidade de escolher a lesão sendo inversamente proporcional ao seu tamanho, para mitigar os efeitos de desbalanceamento de classes para metástases pequenas.

Uma aplicação para a predição de toxicidades na radioterapia feita por \cite{Jalalifar:2022} a partir da união de radiômica com a segmentação de metástases cerebrais realizadas com uma rede de arquitetura formada pela combinação de duas 2D U-Nets, duas 3D U-Nets e uma MSGA, encontrou resultados que foram melhores do que cada uma das redes individuais. Para a primeira 2D U-Net, os \textit{inputs} possuíam tamanho de $512\times512$ \textit{pixels} e $256\times256$ \textit{pixels} para a segunda 2D U-Net. Já para as 3D-Unets, os \textit{voxels} do \textit{input} possuíam tamanho de $128\times128\times128$ \textit{pixels}. Os resultados obtidos incluíram um DSC de $0,90 \pm 0,04$ e uma distância de Hausdorff de $2,3 \pm 0,55$ mm para o conjunto de teste. Os autores também publicaram outro estudo \cite{Jalalifar:2023} considerando apenas as imagens em RM para a predição de efeitos adversos após a radiocirurgia, e utilizaram a mesma arquitetura de rede com valores de BS de 4, 1 e 1 para as redes 2D U-Net, 3D U-Net e MSGA, com respectivas funções perda Dice, \textit{cross-entropy} e a soma das duas. A LR de 0,0001 foi a mesma para as três redes e os resultados para a segmentação foram superiores quando comparados com as redes individuais e uma nnu-Net tida como referência, apresentando resultados para DSC de $0,915 \pm 0,037$ e distância de Hausdorff de $2,1 \pm 0,6$ mm, que também foi superior em relação ao primeiro estudo realizado. Quanto à segmentação realizada nos 5 \textit{follow-ups} após a cirurgia, a rede proposta pelos autores foi superior com exceção do 4º \textit{follow-up}, quando a rede nnU-Net foi menos do que de 1\% melhor nas métricas avaliadas. Além disso, os resultados obtidos nesse segundo estudo foram melhores apenas para a métrica do coeficiente Dice quando comparado ao estudo feito com um \textit{dataset} menor \cite{Jalalifar:2020}, que apresentou uma distância de Hausdorff de $1,12 \pm 0,03$ mm ao ser utilizada uma rede composta por U-Nets 2D e 3D.

\cite{Lee:2022} implementaram uma rede CNN \textit{encoder-decoder} de via dupla para realizar o delineamento de três tipos de tumores cerebrais: schwannoma vestibular, metástases e meningioma. O modelo treinado para delinear metástases cerebrais contou com 100 épocas de treino, com otimizador Adam, e LR de 0,001, com imagens formadas por \textit{patches} de tamanho $256\times256\times20$ \textit{pixels} como \textit{input} e BS de 1. A função perda utilizada foi a Dice generalizada, com fator $\epsilon = 0,00001$. Ao utilizarem os parâmetros de treinos das outras duas patologias no treino para as metástases, foram encontrados resultados otimizados, cujos valores para DSC de $0,85 \pm 0,18$, sensibilidade de $0,87 \pm 0,29$ e precisão de $0,85 \pm 0,18$. Os resultados obtidos para os outros tipos de tumores foram concordantes no que diz respeito a se utilizar os parâmetros de treino cruzados e podem ser importantes para a generabilidade dos modelos treinados.

\cite{Ottesen:2023} realizaram uma comparação entre três diferentes CNNs para a segmentação de metástases cerebrais treinadas com o dataset \textit{BrainMetShare} e testadas em um dataset composto de imagens de ressonância magnética em ponderação $T_1$ \textit{Fast Spin Echo} (SPACE) e FLAIR anotadas por dois radiologistas com 5 e 14 anos de experiência. As arquiteturas de rede utilizadas foram a 2.5D e a 3D HRNetV2, do inglês \textit{high-resolution net V2}, e a nnU-Net, e utilizaram uma mistura de duas funções perda, a \textit{Focal Tversky} e a \textit{binary cross-entropy} . Seus resultados apontam para uma maior eficiência da rede 2.5D em relação à rede 3D e nnU-Net, identificando, respectivamente, 79\%, 71\% e 65\% das metástases. Para a tarefa de segmentação, as redes apresentaram resultados similares, a nnU-Net com DSC de $0,94 \pm 0,05$ e as duas HRNetV2 com DSC de $0,93 \pm 0,04$. Já para o dataset \textit{BrainMetShare}, a eficiência de detecção de metástases foi de, 88\%, 86\% e 78\% com sensibilidades de 0,92, 0,91 e 0,85, respectivamente. Vale ressaltar que a acurácia de detecção de metástases decresce quando essas possuem área de seção menor do que 0,4 cm$^2$.

\cite{Li:2023} utilizaram duas modalidades de imagens em RM, com e sem contraste, e realizaram a subtração entre essas imagens de modo a formar três canais para o \textit{input} em \textit{patches} de tamanho $128\times512\times512$ \textit{pixels}, com \textit{data augmentation} envolvendo rotações, inversões e variação na intensidade aleatórias. A rede utilizada possuía dois estágios, o primeiro sendo constituído de uma rede 3D U-Net para a segmentação de metástases e o segundo uma CNN multi-níveis para a supressão de falsos positivos. Foi utilizado o otimizador Adam, com a união das funções perda \textit{Bias Dice} e \textit{cross-entropy}, durante 500 épocas, com LR de 0,001 e BS de 1. Os resultados obtidos por essa rede foram, em geral, superiores quando comparados às redes SSD, nnU-Net e 3D U-net, sendo que essa última provou-se superior para a segmentação de metástases de diâmetros maiores do que 5 mm.

\begin{table}[ht]
\centering
\caption{Comparação do número N de casos/pacientes no treino o \textit{split} em treino, validação e teste (T/V/T), o coeficiente Dice (DCS), a precisão (PPV), a sensibilidade (TPR) e a quantidade de falsos positivos por paciente (FPP) dos estudos encontrados em literatura para a segmentação de metástases cerebrais. Os melhores resultados estão destacados em negrito.}
\begin{tabular}{ccccccc}
\hline
Autores                         & N                    & T/V/T                 & DCS  & PPV   & TPR   & FPP   \\ \hline
\cite{Liu:2017}                         & 490                  & 5-\textit{fold}                   & 0,67 & -     & -     & -     \\
\cite{Charron:2018}                     & 182                  & 146/18/18                   & 0,79 & -     & \textbf{0,98}  & 4,4   \\
\cite{Hu:2019}                      & 305                  & 245/30/30                    & 0,74  & 0,78   & 0,803   & -   \\
\cite{Grovik:2020}                      & 156                  & 100/5/51                    & 0,8  & 0,8   & 0,5   & 8,3   \\
\cite{Dikici:2020}                     & 158                  & 5-\textit{fold}                   & -    & -     & 0,9   & 9,12  \\
\cite{Xue:2020}                         & 1201                 & 4-\textit{fold}                   & 0,85 & -     & 0,96  & -     \\
\cite{Jalalifar:2020}                     & 106                  & 90/6/10                     & 0,9  & -     & -     & -     \\
\multirow{3}{*}{\cite{Bousabarah:2020}} & \multirow{2}{*}{509} & 469/0/40                    & 0,71 & 0,96  & 0,71  & \textbf{0}     \\
                              &                      & 469/0/40                    & 0,70 & \textbf{1,00}  & 0,71  & 0,05  \\
                              & 257                  & 233/0/24                    & 0,27 & 0,95  & 0,53  & 0,05  \\
\cite{Zhou:2020}                        & 934                  & 748/186                     & 0,8  & 0,6   & 0,9   & 3     \\
\cite{Rudie:2021}                      & 563                  & 413/50/100                    & 0,75  & 0,915   & 0,7   & 0,46   \\

\cite{Pennig:2021}                      & 69                   & 5-\textit{fold}                   & 0,75 & 0,74  & 0,88  & 0,71  \\
\cite{Jünger:2021}                      & 98                   & 66/17/15                    & 0,72 & -     & 0,851 & 1,8   \\
\cite{Park:2021}                        & 188                  & 188/0/94                    & 0,8  & 0,931 & 0,907 & 0,59  \\
\cite{Yi:2021}                          & 156                  & -                           & 0,85 & 0,68  & 0,92  & -     \\
\cite{Yoo:2021}                         & 442                  & 341/36/45                   & 0,62 & 0,75  & 0,74  & 0,53  \\
\cite{Grøvik:2021}                      & 165                  & 100/10/55                   & 0,8  & 0,8   & 0,7   & 12    \\
\cite{Hsu:2021}                         & 511                  & 5-\textit{fold}                    & 0,76 & -     & 0,95  & 2,4   \\
\multirow{3}{*}{\cite{Cho:2021}}        & \multirow{3}{*}{147} & 127/20/35                   & 0,7  & -     & 0,877 & 1,9   \\
                              &                      & 127/20                      & 0,7  & -     & 0,751 & 0,8   \\
                              &                      & 127/20/12                   & 0,8  & -     & 0,947 & 0,5   \\
\cite{Sun:2021}                         & 450                  & 285/72/93                   & -    & -     & -     & -     \\
\cite{Huang:2022}                       & 176                  & 135/-/32                    & 0,81 & 0,987 & 0,975 & -     \\
\cite{Yoo:2022}                         & 65                   & 53/-/12                     & 0,75 & -     & 0,97  & 1,25  \\
\cite{Bouget:2022}                      & 396                  & 5-\textit{fold}                   & 0,9  & 0,9   & 0,9   & 0,061 \\
\cite{Liang:2022}                       & 407                  & 326/78/81                   & 0,73 & -     & 0,91  & 1,9   \\
\cite{Lyu:2022}                         & 1399                 & 10-\textit{fold}                  & 0,9  & -     & 0,9   & -     \\
\multirow{2}{*}{\cite{Liew:2022}}       & 156                  & 84/51/21                    & -    & -     & 0,811 & 2,952 \\
                              & 144                  & 121/23/0                    & -    & -     & 0,74  & 3,131 \\
\cite{Chartrand:2022}                   & 530                  & 383/50/97                   & 0,73 & 0,81  & 0,91  & 0,66  \\
\cite{Jalalifar:2022}                   & 124                  & 89/10/25                    & 0,90 & -     & -     & -     \\
\cite{Lee:2022}                         & 574                  & 474/0/100                   & 0,82 & 0,85  & 0,87  & -     \\
\multirow{3}{*}{\cite{Ottesen:2023}}      & \multirow{3}{*}{221} & \multirow{3}{*}{100/10/116} & 0,93 & -     & 0,88  & 1     \\
                              &                      &                             & 0,93 & -     & 0,86  & 0,4   \\
                              &                      &                             & \textbf{0,94} & -     & 0,78  & 0,1   \\
\cite{Li:2023}                          & 649                  & 295/99/255                  & 0,81 & 0,56  & 0,9   & 5     \\
\cite{Jalalifar:2023}                     & 116                  & 90/-/20                    & 0,92 & -     & -     & -     \\ \hline
\end{tabular}
\fonte{Autoria Própria.}
\label{tab:literatura}
\end{table}
