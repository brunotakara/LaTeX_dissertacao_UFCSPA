% Documento: Cronograma
\chapter{Cronograma}\label{chap:cronograma}

As atividades e suas descrições estão organizadas na seguinte lista e dispostas no cronograma do Quadro \ref{qua:cronograma}.

\begin{enumerate}[I]
    \item Verificar as aplicações de CNNs em datasets neurológicos, bem como o formato das imagens e suas características.
    \item Encontrar e solicitar imagens de metástases cerebrais e OARs em datasets públicos.
    \item Preparar o banco de dados com imagens de metástases cerebrais em RM ou TC, preferencialmente ambos por serem complementares.
    \item Desenvolver o algoritmo para importação de dados e treino da rede neural.
    \item Realizar a segmentação automatizada de metástases cerebrais através do treino da CNN.
    \item Avaliar diferentes estratégias de segmentação, com uso de arquiteturas de rede do estado da arte.
    \item Testar a variação de hiperparâmetros das diferentes arquiteturas e comparar os resultados obtidos.
    \item Realizar a busca de metodologias recém publicadas para aprimoramento do modelo.
    \item Aplicar arquiteturas CNN para escolha daquela com melhores resultados nas métricas definidas.
    \item Realizar os testes da rede neural para identificação e segmentação de metástases cerebrais.
    \item Verificar a concordância das regiões segmentadas através das métricas de avaliação e aprimorar o treino da rede.
    \item Avaliar quantitativamente a segmentação realizada nas ROIs.
    \item Elaborar artigo científico para submissão em revista de Qualis A2, uma vez concluídos os objetivos específicos.
    \item Escrever e revisar a dissertação.
    \item Realizar otimizações no algoritmo em relação ao tempo de operação e resultados das métricas.
\end{enumerate}

% Uma dica para gerar tabelas e quadros é utilizar o Tables Generator, disponível em: https://www.tablesgenerator.com/

\begin{quadro}[ht]
\caption{Cronograma das atividades.\label{qua:cronograma}}
\begin{tabular}{|ccccccccccccc|}
\hline
\multicolumn{13}{|c|}{\textbf{Cronograma}}                    \\ \hline
\multicolumn{13}{|c|}{\textbf{Ano 1}}
\\ \hline
\multicolumn{1}{|c|}{Atividade} & \multicolumn{1}{c|}{Abr} & \multicolumn{1}{c|}{Mai} & \multicolumn{1}{c|}{Jun} & \multicolumn{1}{c|}{Jul} & \multicolumn{1}{c|}{Ago} & \multicolumn{1}{c|}{Set} & \multicolumn{1}{c|}{Out} & \multicolumn{1}{c|}{Nov} & \multicolumn{1}{c|}{Dez} & \multicolumn{1}{c|}{Jan} & \multicolumn{1}{c|}{Fev} & \multicolumn{1}{c|}{Mar} \\ 
\hline
\multicolumn{1}{|c|}{I}         & \multicolumn{1}{c|}{X}    & \multicolumn{1}{c|}{X}    & \multicolumn{1}{c|}{X}    & \multicolumn{1}{c|}{X}    & \multicolumn{1}{c|}{X}    & \multicolumn{1}{c|}{}    & \multicolumn{1}{c|}{}    & \multicolumn{1}{c|}{}    & \multicolumn{1}{c|}{}    & \multicolumn{1}{c|}{}    & \multicolumn{1}{c|}{}    &   \multicolumn{1}{c|}{}  \\ 
\hline
\multicolumn{1}{|c|}{II}        & \multicolumn{1}{c|}{}    & \multicolumn{1}{c|}{X}    & \multicolumn{1}{c|}{X}    & \multicolumn{1}{c|}{}    & \multicolumn{1}{c|}{}    & \multicolumn{1}{c|}{}    & \multicolumn{1}{c|}{}    & \multicolumn{1}{c|}{}    & \multicolumn{1}{c|}{}    & \multicolumn{1}{c|}{}    & \multicolumn{1}{c|}{}    &   \multicolumn{1}{c|}{}  \\ 
\hline
\multicolumn{1}{|c|}{III}       & \multicolumn{1}{c|}{}    & \multicolumn{1}{c|}{}    & \multicolumn{1}{c|}{}    & \multicolumn{1}{c|}{X}    & \multicolumn{1}{c|}{X}    & \multicolumn{1}{c|}{X}    & \multicolumn{1}{c|}{X}    & \multicolumn{1}{c|}{}    & \multicolumn{1}{c|}{}    & \multicolumn{1}{c|}{}    & \multicolumn{1}{c|}{}    &   \multicolumn{1}{c|}{}  \\ \hline
\multicolumn{1}{|c|}{IV}        & \multicolumn{1}{c|}{}    & \multicolumn{1}{c|}{}    & \multicolumn{1}{c|}{}    & \multicolumn{1}{c|}{}    & \multicolumn{1}{c|}{X}    & \multicolumn{1}{c|}{X}    & \multicolumn{1}{c|}{X}    & \multicolumn{1}{c|}{X}    & \multicolumn{1}{c|}{X}    & \multicolumn{1}{c|}{}    & \multicolumn{1}{c|}{}    &  \multicolumn{1}{c|}{}   \\
\hline
\multicolumn{1}{|c|}{V}         & \multicolumn{1}{c|}{}    & \multicolumn{1}{c|}{}    & \multicolumn{1}{c|}{}    & \multicolumn{1}{c|}{}    & \multicolumn{1}{c|}{}    & \multicolumn{1}{c|}{}    & \multicolumn{1}{c|}{}    & \multicolumn{1}{c|}{}    & \multicolumn{1}{c|}{X}    & \multicolumn{1}{c|}{X}    & \multicolumn{1}{c|}{X}    &   \multicolumn{1}{c|}{X} \\
\hline
\multicolumn{1}{|c|}{VI}        & \multicolumn{1}{c|}{}    & \multicolumn{1}{c|}{}    & \multicolumn{1}{c|}{}    & \multicolumn{1}{c|}{}    & \multicolumn{1}{c|}{}    & \multicolumn{1}{c|}{}    & \multicolumn{1}{c|}{}    & \multicolumn{1}{c|}{}    & \multicolumn{1}{c|}{}    & \multicolumn{1}{c|}{X}    & \multicolumn{1}{c|}{X}    &   \multicolumn{1}{c|}{X}  \\ 
\hline
\multicolumn{1}{|c|}{VII}       & \multicolumn{1}{c|}{}    & \multicolumn{1}{c|}{}    & \multicolumn{1}{c|}{}    & \multicolumn{1}{c|}{}    & \multicolumn{1}{c|}{}    & \multicolumn{1}{c|}{}    & \multicolumn{1}{c|}{}    & \multicolumn{1}{c|}{}    & \multicolumn{1}{c|}{}    & \multicolumn{1}{c|}{}    & \multicolumn{1}{c|}{}    &   \multicolumn{1}{c|}{X}  \\ 
\hline
\multicolumn{1}{|c|}{VIII}      & \multicolumn{1}{c|}{}    & \multicolumn{1}{c|}{}    & \multicolumn{1}{c|}{}    & \multicolumn{1}{c|}{}    & \multicolumn{1}{c|}{}    & \multicolumn{1}{c|}{}    & \multicolumn{1}{c|}{}    & \multicolumn{1}{c|}{}    & \multicolumn{1}{c|}{}    & \multicolumn{1}{c|}{}    & \multicolumn{1}{c|}{}    &   \multicolumn{1}{c|}{}  \\ 
\hline
\multicolumn{1}{|c|}{IX}        & \multicolumn{1}{c|}{}    & \multicolumn{1}{c|}{}    & \multicolumn{1}{c|}{}    & \multicolumn{1}{c|}{}    & \multicolumn{1}{c|}{}    & \multicolumn{1}{c|}{}    & \multicolumn{1}{c|}{}    & \multicolumn{1}{c|}{}    & \multicolumn{1}{c|}{}    & \multicolumn{1}{c|}{}    & \multicolumn{1}{c|}{}    &  \multicolumn{1}{c|}{}   \\ 
\hline
\multicolumn{1}{|c|}{X}         & \multicolumn{1}{c|}{}    & \multicolumn{1}{c|}{}    & \multicolumn{1}{c|}{}    & \multicolumn{1}{c|}{}    & \multicolumn{1}{c|}{}    & \multicolumn{1}{c|}{}    & \multicolumn{1}{c|}{}    & \multicolumn{1}{c|}{}    & \multicolumn{1}{c|}{}    & \multicolumn{1}{c|}{}    & \multicolumn{1}{c|}{}    &   \multicolumn{1}{c|}{}  \\ 
\hline
\multicolumn{1}{|c|}{XI}        & \multicolumn{1}{c|}{}    & \multicolumn{1}{c|}{}    & \multicolumn{1}{c|}{}    & \multicolumn{1}{c|}{}    & \multicolumn{1}{c|}{}    & \multicolumn{1}{c|}{}    & \multicolumn{1}{c|}{}    & \multicolumn{1}{c|}{}    & \multicolumn{1}{c|}{}    & \multicolumn{1}{c|}{}    & \multicolumn{1}{c|}{}    &   \multicolumn{1}{c|}{}  \\ 
\hline
\multicolumn{1}{|c|}{XII}       & \multicolumn{1}{c|}{}    & \multicolumn{1}{c|}{}    & \multicolumn{1}{c|}{}    & \multicolumn{1}{c|}{}    & \multicolumn{1}{c|}{}    & \multicolumn{1}{c|}{}    & \multicolumn{1}{c|}{}    & \multicolumn{1}{c|}{}    & \multicolumn{1}{c|}{}    & \multicolumn{1}{c|}{}    & \multicolumn{1}{c|}{}    &   \multicolumn{1}{c|}{}  \\ 
\hline
\multicolumn{1}{|c|}{XIII}      & \multicolumn{1}{c|}{}    & \multicolumn{1}{c|}{}    & \multicolumn{1}{c|}{}    & \multicolumn{1}{c|}{}    & \multicolumn{1}{c|}{}    & \multicolumn{1}{c|}{}    & \multicolumn{1}{c|}{}    & \multicolumn{1}{c|}{}    & \multicolumn{1}{c|}{}    & \multicolumn{1}{c|}{}    & \multicolumn{1}{c|}{}    &   \multicolumn{1}{c|}{}  \\
\hline
\multicolumn{1}{|c|}{XIV}        & \multicolumn{1}{c|}{}    & \multicolumn{1}{c|}{}    & \multicolumn{1}{c|}{}    & \multicolumn{1}{c|}{}    & \multicolumn{1}{c|}{}    & \multicolumn{1}{c|}{}    & \multicolumn{1}{c|}{}    & \multicolumn{1}{c|}{}    & \multicolumn{1}{c|}{}    & \multicolumn{1}{c|}{}    & \multicolumn{1}{c|}{X}    &  \multicolumn{1}{c|}{X}   \\ 
\hline
\multicolumn{1}{|c|}{XV}         & \multicolumn{1}{c|}{}    & \multicolumn{1}{c|}{}    & \multicolumn{1}{c|}{}    & \multicolumn{1}{c|}{}    & \multicolumn{1}{c|}{}    & \multicolumn{1}{c|}{}    & \multicolumn{1}{c|}{}    & \multicolumn{1}{c|}{}    & \multicolumn{1}{c|}{}    & \multicolumn{1}{c|}{}    & \multicolumn{1}{c|}{}    &   \multicolumn{1}{c|}{}  \\ 
\hline
\multicolumn{13}{|c|}{\textbf{Ano 2}}                         \\ \hline
\multicolumn{1}{|c|}{Atividade} & \multicolumn{1}{c|}{Abr} & \multicolumn{1}{c|}{Mai} & \multicolumn{1}{c|}{Jun} & \multicolumn{1}{c|}{Jul} & \multicolumn{1}{c|}{Ago} & \multicolumn{1}{c|}{Set} & \multicolumn{1}{c|}{Out} & \multicolumn{1}{c|}{Nov} & \multicolumn{1}{c|}{Dez} & \multicolumn{1}{c|}{Jan} & \multicolumn{1}{c|}{Fev} & \multicolumn{1}{c|}{Mar} \\ 
\hline
\multicolumn{1}{|c|}{I}         & \multicolumn{1}{c|}{}    & \multicolumn{1}{c|}{}    & \multicolumn{1}{c|}{}    & \multicolumn{1}{c|}{}    & \multicolumn{1}{c|}{}    & \multicolumn{1}{c|}{}    & \multicolumn{1}{c|}{}    & \multicolumn{1}{c|}{}    & \multicolumn{1}{c|}{}    & \multicolumn{1}{c|}{}    & \multicolumn{1}{c|}{}    &   \multicolumn{1}{c|}{}  \\ 
\hline
\multicolumn{1}{|c|}{II}        & \multicolumn{1}{c|}{}    & \multicolumn{1}{c|}{}    & \multicolumn{1}{c|}{}    & \multicolumn{1}{c|}{}    & \multicolumn{1}{c|}{}    & \multicolumn{1}{c|}{}    & \multicolumn{1}{c|}{}    & \multicolumn{1}{c|}{}    & \multicolumn{1}{c|}{}    & \multicolumn{1}{c|}{}    & \multicolumn{1}{c|}{}    &  \multicolumn{1}{c|}{}   \\ 
\hline
\multicolumn{1}{|c|}{III}       & \multicolumn{1}{c|}{}    & \multicolumn{1}{c|}{}    & \multicolumn{1}{c|}{}    & \multicolumn{1}{c|}{}    & \multicolumn{1}{c|}{}    & \multicolumn{1}{c|}{}    & \multicolumn{1}{c|}{}    & \multicolumn{1}{c|}{}    & \multicolumn{1}{c|}{}    & \multicolumn{1}{c|}{}    & \multicolumn{1}{c|}{}    &   \multicolumn{1}{c|}{}  \\ 
\hline
\multicolumn{1}{|c|}{IV}        & \multicolumn{1}{c|}{}    & \multicolumn{1}{c|}{}    & \multicolumn{1}{c|}{}    & \multicolumn{1}{c|}{}    & \multicolumn{1}{c|}{}    & \multicolumn{1}{c|}{}    & \multicolumn{1}{c|}{}    & \multicolumn{1}{c|}{}    & \multicolumn{1}{c|}{}    & \multicolumn{1}{c|}{}    & \multicolumn{1}{c|}{}    &   \multicolumn{1}{c|}{}  \\ 
\hline
\multicolumn{1}{|c|}{V}         & \multicolumn{1}{c|}{X}    & \multicolumn{1}{c|}{X}    & \multicolumn{1}{c|}{}    & \multicolumn{1}{c|}{}    & \multicolumn{1}{c|}{}    & \multicolumn{1}{c|}{}    & \multicolumn{1}{c|}{}    & \multicolumn{1}{c|}{}    & \multicolumn{1}{c|}{}    & \multicolumn{1}{c|}{}    & \multicolumn{1}{c|}{}    &   \multicolumn{1}{c|}{}  \\ 
\hline
\multicolumn{1}{|c|}{VI}        & \multicolumn{1}{c|}{X}    & \multicolumn{1}{c|}{X}    & \multicolumn{1}{c|}{}    & \multicolumn{1}{c|}{}    & \multicolumn{1}{c|}{}    & \multicolumn{1}{c|}{}    & \multicolumn{1}{c|}{}    & \multicolumn{1}{c|}{}    & \multicolumn{1}{c|}{}    & \multicolumn{1}{c|}{}    & \multicolumn{1}{c|}{}    &   \multicolumn{1}{c|}{}  \\ 
\hline
\multicolumn{1}{|c|}{VII}       & \multicolumn{1}{c|}{X}    & \multicolumn{1}{c|}{X}    & \multicolumn{1}{c|}{X}    & \multicolumn{1}{c|}{}    & \multicolumn{1}{c|}{}    & \multicolumn{1}{c|}{}    & \multicolumn{1}{c|}{}    & \multicolumn{1}{c|}{}    & \multicolumn{1}{c|}{}    & \multicolumn{1}{c|}{}    & \multicolumn{1}{c|}{}    &  \multicolumn{1}{c|}{}   \\ 
\hline
\multicolumn{1}{|c|}{VIII}      & \multicolumn{1}{c|}{}    & \multicolumn{1}{c|}{}    & \multicolumn{1}{c|}{}    & \multicolumn{1}{c|}{X}    & \multicolumn{1}{c|}{X}    & \multicolumn{1}{c|}{X}    & \multicolumn{1}{c|}{}    & \multicolumn{1}{c|}{}    & \multicolumn{1}{c|}{}    & \multicolumn{1}{c|}{}    & \multicolumn{1}{c|}{}    &   \multicolumn{1}{c|}{}  \\ 
\hline
\multicolumn{1}{|c|}{IX}        & \multicolumn{1}{c|}{}    & \multicolumn{1}{c|}{}    & \multicolumn{1}{c|}{}    & \multicolumn{1}{c|}{}    & \multicolumn{1}{c|}{}    & \multicolumn{1}{c|}{}    & \multicolumn{1}{c|}{X}    & \multicolumn{1}{c|}{X}    & \multicolumn{1}{c|}{}    & \multicolumn{1}{c|}{}    & \multicolumn{1}{c|}{}    &  \multicolumn{1}{c|}{}   \\
\hline
\multicolumn{1}{|c|}{X}         & \multicolumn{1}{c|}{}    & \multicolumn{1}{c|}{}    & \multicolumn{1}{c|}{}    & \multicolumn{1}{c|}{}    & \multicolumn{1}{c|}{}    & \multicolumn{1}{c|}{}    & \multicolumn{1}{c|}{}    & \multicolumn{1}{c|}{X}    & \multicolumn{1}{c|}{X}    & \multicolumn{1}{c|}{}    & \multicolumn{1}{c|}{}    &   \multicolumn{1}{c|}{}  \\ 
\hline
\multicolumn{1}{|c|}{XI}        & \multicolumn{1}{c|}{}    & \multicolumn{1}{c|}{}    & \multicolumn{1}{c|}{}    & \multicolumn{1}{c|}{}    & \multicolumn{1}{c|}{}    & \multicolumn{1}{c|}{}    & \multicolumn{1}{c|}{X}    & \multicolumn{1}{c|}{X}    & \multicolumn{1}{c|}{X}    & \multicolumn{1}{c|}{}    & \multicolumn{1}{c|}{}    &   \multicolumn{1}{c|}{}  \\ 
\hline
\multicolumn{1}{|c|}{XII}       & \multicolumn{1}{c|}{}    & \multicolumn{1}{c|}{}    & \multicolumn{1}{c|}{}    & \multicolumn{1}{c|}{}    & \multicolumn{1}{c|}{}    & \multicolumn{1}{c|}{}    & \multicolumn{1}{c|}{X}    & \multicolumn{1}{c|}{X}    & \multicolumn{1}{c|}{X}    & \multicolumn{1}{c|}{}    & \multicolumn{1}{c|}{}    &  \multicolumn{1}{c|}{}   \\ 
\hline
\multicolumn{1}{|c|}{XIII}      & \multicolumn{1}{c|}{}    & \multicolumn{1}{c|}{}    & \multicolumn{1}{c|}{}    & \multicolumn{1}{c|}{}    & \multicolumn{1}{c|}{}    & \multicolumn{1}{c|}{}    & \multicolumn{1}{c|}{}    & \multicolumn{1}{c|}{X}    & \multicolumn{1}{c|}{X}    & \multicolumn{1}{c|}{X}    & \multicolumn{1}{c|}{X}    &  \multicolumn{1}{c|}{}   \\ 
\hline
\multicolumn{1}{|c|}{XIV}        & \multicolumn{1}{c|}{X}    & \multicolumn{1}{c|}{X}    & \multicolumn{1}{c|}{X}    & \multicolumn{1}{c|}{X}    & \multicolumn{1}{c|}{X}    & \multicolumn{1}{c|}{X}    & \multicolumn{1}{c|}{X}    & \multicolumn{1}{c|}{X}    & \multicolumn{1}{c|}{X}    & \multicolumn{1}{c|}{X}    & \multicolumn{1}{c|}{X}    &   \multicolumn{1}{c|}{X}  \\ 
\hline
\multicolumn{1}{|c|}{XV}         & \multicolumn{1}{c|}{}    & \multicolumn{1}{c|}{}    & \multicolumn{1}{c|}{}    & \multicolumn{1}{c|}{}    & \multicolumn{1}{c|}{}    & \multicolumn{1}{c|}{}    & \multicolumn{1}{c|}{}    & \multicolumn{1}{c|}{}    & \multicolumn{1}{c|}{}    & \multicolumn{1}{c|}{X}    & \multicolumn{1}{c|}{X}    &   \multicolumn{1}{c|}{X}  \\ 
\hline
\end{tabular}
\fonte{Autoria própria}
\end{quadro}

